\begin{center}
    \Large{\textbf{Аннотация}}
\end{center}

Широкое внедрение систем машинного обучения в масштабах общества требует глубокого понимания долгосрочных последствий, которые эти системы оказывают на окружающую среду, включая потерю доверия, усиление ошибок и нарушение требований безопасности искусственного интеллекта.
В данной работе вводится процесс многократного обучения для совместного описания нескольких явлений, таких как петли обратной связи (feedback loop), усиление ошибок (error amplification), дрейф понятий (concept drift), эхо-камеры (echo chambers) и другие. Этот процесс включает в себя весь цикл получения данных, обучения прогностической модели и предоставления прогнозов конечным пользователям в рамках единой математической модели.
Отличительной особенностью многократного обучения является то, что состояние среды с течением времени становится зависимым от результатов работы модели, что нарушает обычные предположения о распределении данных.
В данной работе вводится новая модель динамических систем процесса многократного обучения и находится предельное множество функций плотности вероятностей данных для режимов работы системы с положительной и отрицательной петлей обратной связи.
Также проводится серия вычислительных экспериментов на задаче обучения с учителем на двух синтетических наборах данных. Экспериментальные данные соответствуют теоретическим предсказаниям, полученным на основе динамической модели. Результаты данной работы демонстрируют применимость предложенного подхода к изучению процессов многократного обучения в системах искусственного интеллекта и открывают широкие возможности для дальнейших исследований в этой области.