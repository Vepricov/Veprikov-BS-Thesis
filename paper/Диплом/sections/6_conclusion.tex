\section{Заключение}

Исследуемая проблема многократного машинного обучения важна, поскольку алгоритм искусственного интеллекта почти всегда является частью более крупной программной системы и взаимодействует с ее окружением. Важно понять, как происходит это взаимодействие, как оно изменяет качественные характеристики системы машинного обучения и есть ли нарушения требований достоверности.

В данной работе решается задача математического моделирования. В качестве объекта исследования выступает система машинного обучения, которой ставится в соответствие динамическая система, определяемая эволюционным отображением $\text{D}_t$ \eqref{system}. Теорема~\ref{delta} дает понимание того, как это отображение преобразует начальное распределение данных, и что является предельным распределением: дельта-функция или нулевое распределение. На основании этого результата можно судить о том, существует ли положительная петля обратной связи и порождается смещение данных, или же присутствует отрицательная петля обратной связи и ухудшается качество прогнозов. В качестве измеряемых величин выступают значения функций плотности невязок модели на шаге $t$ в точке $0$: $\{f_t(0)\}_{t = 0}^\infty$. Именно по поведению этой последовательности, согласно Теореме \ref{delta}, можно понять какой будет предел у системы \eqref{system}.

Теорема~\ref{semigroup} является критерием автономности нашей динамической системы, полезным свойством для дальнейшего анализа. Лемма~\ref{moments} говорит о стремлении к нулю моментов невязок в петле положительной обратной связи. Эти результаты, в отличие от утверждений Теоремы~\ref{delta}, легче проверить экспериментально, что делает их полезными в практических приложениях.

Было исследовано влияние преобразования признаков на предельное множество динамической системы, описывающей процесс многократного машинного обучения. Были сформулированы и доказаны достаточные условия для его сохранения в виде Лемм \ref{loss} и \ref{bvd}. Полученные результаты можно применить для исследования сложных систем с помощью простых моделей. Таким образом, повышается значимость результатов \cite{veprikov2024mathematical, veprikov2023matematicheskaya, khritankov2021hidden}, где исследованы системы многократного машинного обучения с моделями линейной регрессии и бустингом над случайным лесом. Из наличия в таких системах эффектов положительной обратной связи, при выполнении условий Леммы \ref{loss} или \ref{bvd}, будет следовать их возникновение и при использовании более продвинутых моделей, например, глубоких нейронных сетей.

В данной работе разработана имитационная модель процесса многократного обучения и продемонстрированы эффекты многократного машинного обучения в двух постановках: скользящее окно и обновление выборки (Рис.~\ref{ex_set}). Обе постановки можно использовать для любых других экспериментов с многократным машинным обучением. Как показано в эксперименте~\ref{exp_4}, постановка обновление выборки является автономной, а <<скользящее окно>> -- нет.

Все полученные результаты строго доказаны и проверены в вычислительных экспериментах. Предложенные гипотезы оказались верными, практические результаты согласуются с теорией. 

По итогам работы была подана статья в Q1 журнал \cite{veprikov2024mathematical}, рассказано выступление на конференции ММРО-2023 \cite{veprikov2023matematicheskaya}, материалы которой были поданы в журнал <<Искусственный интеллект и принятие решений>>. Вклад автора диплома состоит в доказательстве теоретических исследований, участии в постановке экспериментов, сборе, обработке и анализе их результатов.

Развитие идей, изложенных в данной дипломной работе, важно для разработки систем машинного обучения и их приложений. Будущие исследования могут включать больше экспериментов на реальных наборах данных и более сложных моделях. 
В экспериментах и при обсуждении Теоремы~\ref{delta} не рассматривается вывод огибающей функции $g$. Возможно, ее построению следует посвятить будущие работы. Другим направлением может служить наделение множества $\textbf{F}$ метрикой, например расстоянием Кульбака-Лейблера, и анализ отображений $\text{D}_t$ в этом метрическом пространстве.

